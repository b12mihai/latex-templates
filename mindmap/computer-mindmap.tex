% Author: Mihai Barbulescu
% Source: It is based on the example from http://www.texample.net/tikz/examples/computer-science-mindmap/

\documentclass{article}
\usepackage[landscape]{geometry}
\usepackage{tikz}
\usetikzlibrary{mindmap,trees}
\usepackage{verbatim}

\usepackage[utf8x]{inputenc}		% diacritice
\usepackage[russian,ngerman,romanian]{babel} 
\title{Aplicația 6 - Harta Conceptuală}
\author{Bărbulescu Mihai \\ 331CA}

\begin{document}
\maketitle
\begin{tikzpicture}
  \path[mindmap,concept color=black,text=white]
    node[concept] {Calculator}
    [clockwise from=10]
    
    child[concept color=green!50!black] {
      node[concept] {Software Free and Open Source}
      [clockwise from=100]
      child { node[concept] {KiCad} }
      child { node[concept] {GNOME Evince} }
      child { node[concept] {Libre Office} }
      child { node[concept] {\LaTeX} }
      child { node[concept] {Linux} }
      child { node[concept] {Pidgin} }
    }  
    child[concept color=blue] {
      node[concept] {IAC}
      [clockwise from=-30]
      child { node[concept] {strategie didactică} }
      child { node[concept] {profesor modern} }
    }
    %child[concept color=red] { node[concept] {technical} }
    child[concept color=cyan!70!black] {
	node[concept] {Internet}
	[clockwise from=-60]
	child { 
	  node[concept] {Google}
	  [clockwise from=-30]
	  child { node[concept] {Google Docs} }
	  child { node[concept] {Google Maps} }
	  child { node[concept] {Google Scholar} }
	}
	child { 
	  node[concept] {Aplicații Web 2.0}
	  [clockwise from=-50]
	  child { node[concept] {Facebook} }
	  child { node[concept] {Twitter} }
	  child { node[concept] {Wordpress} }
	  child { node[concept] {Zoho} }
	  child { node[concept] {Wikipedia} }
	}
	child { node[concept] {Rețelistică} }
    }
    child[concept color=orange] { 
      node[concept] {Computer Science} 
      [clockwise from=180]
	child { 
	  node[concept] {Hardware}
	  [clockwise from=180]
	  child { node[concept] {Procesoare} }
	  child { node[concept] {Placă video} }
	  child { node[concept] {Tranzistor} }
	}
	child { 
	  node[concept] {Calcul paralel și distribuit}
	}
	child { 
	  node[concept] {Algoritmi}
	}
    };
\end{tikzpicture}

\end{document}